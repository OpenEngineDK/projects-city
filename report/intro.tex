% -*- mode: latex; mode: auto-fill; -*-
\section{Introduction}

Rendering of natural, pleasant looking shadows is of great importance
when one wants to create realistic images in computer
graphics. Shadows improves spatial perception and greatly contributes
to the overall aestetics of the image. In general shadow calculations
approximates how light originating from various sources is occluded by
the objects in the scene. Several raytracing-based methods exist,
but these are often computationally expensive and as such not feasible
for real-time rendering. 

In this report we present two different methods for approximating
shadows. Both techniques are popular and widely used within the gaming
industry, due to their computational efficiency and at the same time
impressive visual results. First we look at a technique known as
ambient occlusion which approximates soft shadows caused by occluding
geometry. This technique gives a finer perception of the curvature of
surfaces. We then look at shadow maps which are used for producing
more noticable shadows that occur, when an object is blocking a
light source emitting light in a particular direction.

%%% Local Variables: 
%%% mode: latex
%%% TeX-master: "master"
%%% End: 
