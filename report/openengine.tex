\section{OpenEngine}

OpenEngine is a multipurpose, real-time 3D engine. Its development
started as part of a course in game engine design
(\url{http://aula.au.dk//courses/CGDF07/}). Since then it have been
used for everything from masters thesis and study groups to exhibition
on artificial insemination at the Steno Museum.

In this project, we are using, among other, the following components
of OpenEngine:

\begin{itemize}
\item The \textbf{Scene Graph}. This makes handling transformations
  and structuring easy.
\item The \textbf{OpenGL Renderer}. We have based our shadow map and
  ambient occlusion implementations on the existing renderer.
\item \textbf{File Loading}. OpenEngine have extensions for a lot of
  different model format, including Collada and the Stanford PLY format.
\item \textbf{Camera} and \textbf{Light} abstraction. These enabled us
  to work with light and camera settings easy.
\item \textbf{Inspection Bars}. These are handy tools to change
  settings using a on screen display.
\end{itemize}

We would like to stress that this report is not about OpenEngine and
as such we do not cover the details of the various features used. This
report is mainly about the particular shadow algorithms and in
section \ref{sec:files} we highlight the files containing the relevant
C++ and GLSL shader code. These files can to a large extent be viewed
independently from OpenEngine, as we use a lot of custom OpenGL code to
do most of the work.

%%% Local Variables: 
%%% mode: latex
%%% TeX-master: "master"
%%% End: 
