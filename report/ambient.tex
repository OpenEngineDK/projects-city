% -*- mode: latex; mode: auto-fill; -*-
\section{Ambient Occlusion}

Ambient Occlusion is a popular technique for approximating soft
shadows in a scene. The technique deals only with the kind of light
that is scattered in all directions (ambient light) and as such no
particular light source is taken into consideration. In other words
ambient occlusion assumes a global illumination of uniform intensity
and is completely independent of light sources.
% It is assumed that the ambient light is of equal intensity
% throughout the scene.
The intensity of the ambient light on a particular point on a
three-dimensional surface is determined by the overall occlusion due
to geometrical objects in the vicinity of the point. In general the
ambient occlusion of a three-dimensional point $\mathbf{P}$ with
surface normal $\hat{n}$ can be described by the following equation
\citep{shanmugam}:
\begin{equation}
  \label{eq:ao}
  A(\mathbf{P},\hat{n}) = \frac{1}{\pi} \int_\Omega V(\hat{\omega},\mathbf{P})\max(\hat{\omega}\cdot\hat{n},0)\,\mathrm{d}\hat{\omega}
\end{equation}
where we integrate over every direction within a fixed radius
hemisphere $\Omega$ centered in the point $\mathbf{P}$ with the zenith
axis pointing in the direction of the surface normal $\hat{n}$. $V$ is
a function that returns $1$ if any geometrical object is intersected
when tracing from $\mathbf{P}$ in the direction of $\hat{\omega}$ and
$0$ otherwise. The resulting value lies on the interval $[0,1]$ and
represents a weighted average of every direction within the
hemisphere. Each weight is dependent on the angle between the normal
$\hat{n}$ and the direction $\hat{\omega}$ and ensures that object
intersections with large angles contributes with less occlusion than
intersections in a direction which is closer to the direction of the
surface normal.

For better results one can extend equation (\ref{eq:ao}) with an
attenuation function to account for the distance between $\mathbf{P}$
and an the occluding object. This would yield
\begin{equation}
  \label{eq:ao_att}
  A(\mathbf{P},\hat{n}) = \frac{1}{\pi} \int_\Omega
  V(\hat{\omega},\mathbf{P})\max(\hat{\omega}\cdot\hat{n},0)W(R,r) \,\mathrm{d}\hat{\omega}
\end{equation}
where $R$ is the hemisphere radius and $r$ is the distance between
$\mathbf{P}$ and the occluding object. For instance $W$ could be
defined as the quadratic attenuation function scaled by the hemisphere
radius given by
\begin{equation}
  \label{eq:quad_att}
  W(R,r) = \frac{1-r^2}{R}
\end{equation}

\subsection{Screen Space Ambient Occlusion}
Ambient occlusion as described in the above could be done by tracing a
finite number of rays for a discretized set of points on each planar
surface and performing intersection tests with nearby
geometry. Obviously this is a computationally expensive approach which
would not work in real-time rendering. Also it would be wasteful to
perform the calcutions described in (\ref{eq:ao}) and
(\ref{eq:ao_att}) on every planar surface in the scene since only a
very small number surfaces are actually visible in the final image.

Screen space ambient occlusion \citep{ssao} is a fast technique for
approximating ambient occlusion as previously described. The algorithm
is completely independent of scene geometry as it uses the depth
buffer as a rough approximation of the visible geometry. Assuming that
the depth buffer is available together with a buffer containing the
surface normals given in screen space coordinates of the geometry
behind each fragment, this technique can be viewed as a geometry
independent post-processing effect. As a result the complexity of the
algorithm only depends on the resolution of the screen.

The basic idea behind screen space ambient occlusion is that the depth
buffer can be viewed as a height map depicting the curvature of
all the front-most visible geometry with respect to the eye
position. Each two-dimensional position on the screen surface
can then be translated into a three-dimensional point in screen space by
means of the depth value. A finite number of rays within the hemisphere
in the direction of the surface normal can then be traced using a
technique called ray-marching. Ray-marching traces a ray by stepping
along the direction of the ray. The step size is a fixed predermined
value. Occluders are then detected by testing intersections with the
height map determined by the depth buffer.

\subsection{Horizon Based Ambient Occlusion}
A further performance improvement is achieved by realising that
ray-marching in eye-space is unnecessary since our input data is
two-dimensional \citep{hbao}. This is the underlying observation of
the horizon based ambient occlusion approach, where we trace entirely
in image space. 

\subsection{Implementation}
Impl

\subsection{Results}
Demo!

%%% Local Variables: 
%%% mode: latex
%%% TeX-master: "master"
%%% End: 
