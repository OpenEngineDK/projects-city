
\section{Shadow Mapping}

Shadow Mapping is a image-based shadowing technique.

The technique was first described in \citep{lance78}, and have since
then been using quite a lot.

In 1985, Pixar used shadow mapping to create the video Luxo Jr. This
non real-time video was extremely impressive at its time. Especially
because of the shadows. 16 years later, in 2001 at Macworld Expo
Japan, Steve Jobs shows a real-time version of Luxo Jr. with the
exact same scene rendered using OpenGL.

In figure \ref{fig:scene} a small scene rendered without shadows can
be seen. Later when shadow mapping have been added, it is clear to see
how much difference simple shadows make.

\begin{figure}[h]
  \centering
  \includegraphics[width=\textwidth]{gfx/scenenoshadow}  
  \caption{Scene Without Shadows}
  \label{fig:scene}
\end{figure}

Shadow Mapping is a two-pass algorithm. The first step is to create
the ``Shadow Map'' which describes the scene from the light sources
point of view. In the second pass this map will be used to add shadows
while rendering the scene as seen by the camera.


\subsection{The light source}

To create the shadow map, the light sources need a bit more
information than usually. Regular lights in OpenGL have properties
that describes the amount of light emitted, and the placement of the
light source. This is not enough, as we need to create a project to be
able to render the scene from the light. \todoPtx{omskriv?}

Because of this, a light sources is more like a camera, which gives us
a nice abstraction. \todoPtx{complete me}


\subsection{Creating the Shadow map}

The first part of the algorithm is to create the shadow map. This is
done by rendering the scene using the light source as the camera, and
extracting the depth buffer.

\todoPtx{Husk at implementere det} The only interesting information is
the depth, so the rendering can be speed up be disabling textures,
lighting, colors and more. Its also convenient to disable front faces
to avoid self-shadowing. % bla bla?

\begin{figure}[h]
  \centering
  \includegraphics[width=\textwidth]{gfx/shadowmap}  
  \caption{Shadow Map}
  \label{fig:shadowmap}
\end{figure}

In figure \ref{fig:shadowmap}, the shadow map for the scene in figure
\ref{fig:scene} is shown.

\subsection{Rendering the Scene}\label{sec:render}


The next step is to render the scene with shadows. For each fragment
in the finished scene, we must decide if it is in a shadow or
not. This is done by comparing the depth of the fragment with depth in
the shadow map.

\subsubsection*{Lookup in the shadow map}

Before we can compare anything, we need to find the corresponding
point in the shadow map. To calculate this, we need create a
projection matrix that converts model coordinates into projective
texture coordinates.

The first step is to convert model space to world space, this is done
using the regular modeling matrix ($M$)that we also use for rendering
the scene. The next step is to convert this to light space, using the
lights view matrix ($V_{light}$). We convert this to screen space via
the lights projection matrix ($P_{light}$). This will give us a
coordinate in the $[-1,1] \times [-1,1]$ range. This is a problem, as
texture space is in the range $[0,1] \times [0,1]$, so we add a bias
matrix ($B$), that does this conversion.

\begin{align*}
  B &= \begin{bmatrix}
    \frac{1}{2} & 0   & 0   & \frac{1}{2} \\
    0   & \frac{1}{2} & 0   & \frac{1}{2} \\
    0   & 0   & \frac{1}{2} & \frac{1}{2} \\
    0   & 0   & 0   & 1   \\
  \end{bmatrix} \\
  T &= BP_{light}V_{light}M
\end{align*}

\subsubsection*{Render shadow}

Once we have the right point, all there is left is to compare the
depth of the shadow map, and the coordinates $z$-component. When the
value in the shadow map is largest, it means that something is closer
to the light source, than the fragment we are processing.

\todoPtx{Kunne skrives bedre...}
Now we know if the fragment is in shadow, and change the color
to visualise it.

\begin{figure}[h]
  \centering
  \includegraphics[width=\textwidth]{gfx/scenewithshadow}  
  \caption{Scene with shadows}
  \label{fig:sceneshadow}
\end{figure}

In figure \ref{fig:sceneshadow}, the scene is rendered using a single
shadow map.

\newpage
%% -------------


\subsection{Implementation}

Our implementation of shadow mapping is based on OpenGL 2.1 with
shader language 1.20, and the framebuffer extension
(\texttt{GL\_ARB\_framebuffer\_object}).

Shadow mapping could be implemented on older hardware without
shaders \citep{nv01}, but that requires specific extensions, but are
outside the topic of this course.


\subsubsection*{Setup}

The first step is to setup a framebuffer object (FBO) and a texture to
contain the shadow map.

Using the buildin support for FBO in OpenEngine, this becomes quite elegant:

\begin{cppcode}
void ShadowMap::generateShadowFBO(RenderingEventArg arg) {
	int shadowMapWidth = SHADOW_MAP_RATIO * arg.canvas.GetWidth();
	int shadowMapHeight = SHADOW_MAP_RATIO * arg.canvas.GetHeight();
    IRenderer& renderer = arg.renderer;
    Vector<2,int> dims(shadowMapWidth,shadowMapHeight);
    fb = new FrameBuffer(dims,0,true);
    renderer.BindFrameBuffer(fb);
}
\end{cppcode}

This code creates a object that represents a FBO and its attached
textures. The arguments (\texttt{0,true}) creates a FBO with zero color
attachments and a texture as the depth attachment.

The call to \texttt{renderer.BindFrameBuffer(fb)} will create the
necessary OpenGL calls and load both FBO and its corresponding
textures to the driver.

\subsubsection*{Shadow Map}

To create the shadow map, we bind the FBO, and draw the scene from the
light sources point of view. 

\begin{cppcode}
void ShadowMap::MakeMap(RenderingEventArg arg) {
     // only a snippet
    glEnable(GL_DEPTH_TEST);
    glEnable(GL_CULL_FACE);
    glBindFramebufferEXT(GL_FRAMEBUFFER_EXT,fb->GetID());
    glUseProgramObjectARB(0);
    glViewport(0, 0,
               w * SHADOW_MAP_RATIO,
               h * SHADOW_MAP_RATIO);
    glClear( GL_DEPTH_BUFFER_BIT);
    glColorMask(GL_FALSE, GL_FALSE, GL_FALSE, GL_FALSE); 
    glEnable(GL_POLYGON_OFFSET_FILL);
    glPolygonOffset(poff, punits);
    arg.renderer.ApplyViewingVolume(*(light->lightCam));
    glCullFace(GL_FRONT);
    arg.canvas.GetScene()->Accept(*this); return;
    glDisable(GL_POLYGON_OFFSET_FILL);
}
\end{cppcode}

Here we bind the FBO, removes any shader, and setup the viewport to
the size of our shadow map. We also enable culling of front faces and
set up polygon offset to avoid self shadowing.

The \texttt{ApplyViewingVolume} method takes a camera, and send both
modelview and projection to OpenGL.

% noget om visitor?

\subsubsection*{Rendering shadows}

As explained in section \ref{sec:render}, we need to construct a
matrix projection to make our lookup in the shadow map. We need to
access the matrix from our shader, so we load it into the texture
matrix. To avoid overriding any other texture transformations, we
select the last texture unit, and hope no one is using it.

\begin{cppcode}
  void ShadowMap::SetTextureMatrix() {    
    IViewingVolume* vv = light->lightCam;
    Matrix<4,4,float> B(.5, .0, .0,  .0,
                        .0, .5, .0,  .0,
                        .0, .0, .5,  .0,
                        .5, .5, .5, 1.0);
    Matrix<4,4,float> Vl = vv->GetViewMatrix();
    Matrix<4,4,float> Pl = vv->GetProjectionMatrix();    
    Matrix<4,4,float> T = Vl*Pl*B; // The order is reversed T = B*Pl*Vl 
                                   // M is implicit (we add M in VisitTransformationNode)
    float T_arr[16];
    T.ToArray(T_arr);

    glMatrixMode(GL_TEXTURE);
    glActiveTextureARB(GL_TEXTURE7);
    glLoadMatrixf(T_arr);

    // Go back to normal matrix mode
    glMatrixMode(GL_MODELVIEW);
}
\end{cppcode}

When this is done, we just need to render the scene to screen using a
shadow map shader.

In the vertex shader, the only thing we need to do, is multiply the
vertex with our transformation matrix.

\begin{cppcode}
varying vec4 ShadowCoord;

void main(void) {
    ShadowCoord = gl_TextureMatrix[7] * gl_Vertex;
    gl_Position = ftransform();
}
\end{cppcode}

In the fragment we do the lookup, and changes the color if the
fragment is in shadow.

\begin{cppcode}  
uniform sampler2DShadow ShadowMap;
uniform float ShadowAmount;
varying vec4 ShadowCoord;

float lookup(float x, float y) {
    float d = shadow2DProj(ShadowMap,ShadowCoord + vec4(x,y,0,0)).r;
    return d < 1.0 ? ShadowAmount : 1.0;
}

void main() {
    vec4 color;

    color = gl_Color;

    float shadow = lookup(0.0, 0.0);
    color *= shadow;

    color.a = 1.0;
    gl_FragColor = color;
}

\end{cppcode}

This is the most simple version, where we use \texttt{shadow2DProj} to
find and compare the depth value in the shadow map. if
\texttt{shadow2DProj} returns 1.0, it means that the fragment is lit,
otherwise its in shadow.

\texttt{lookup} takes an offset, so we can use the function to make
anti-aliasing. We do this by looking at the neighboring pixels, and
dividing by the number of pixels we looked at.

\begin{cppcode}
    float sd=0.017;
    float shadow = lookup(0.0, 0.0);
    shadow += lookup( sd,  sd);
    shadow += lookup( sd, -sd);
    shadow += lookup(-sd,  sd);
    shadow += lookup(-sd, -sd);
    shadow += lookup(0.0, -sd);
    shadow += lookup(0.0,  sd);
    shadow += lookup(-sd, 0.0);
    shadow += lookup( sd, 0.0);
    shadow /= 9.0;
\end{cppcode}

This gives more smooth edges on the shadows.

\subsubsection*{Lighting}

When we add vertex and fragment shaders to our pipeline, we loose all
the fixed functions, so we need to implement our own lighting. 

\todoPtx{skriv om light?}

\subsection{Limitations}

Shadow mapping is a technique for adding shadows in projection based \todoPtx{fixy?}
3D rendering. As such, it is limited by what is possible with projections.

One of these is that we cant have a field of view larger than 180
Therefor shadow mapping wont work with point light sources if they are
placed between objects. A way to fix this would be to render multiple
shadow maps per light source, but that makes it more expensive.

\todoPtx{more limits}

\subsection{Future Work}

This implementation only supports one light source, having more would
be a easy and useful extension.


%%% Local Variables: 
%%% mode: latex
%%% TeX-master: "master"
%%% End: 
