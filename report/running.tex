
\section{Running and Files}

The code have been tested on \texttt{faugeras}, and should compile and
run using the \texttt{MAKE} and \texttt{RUN} files. To build and run
on a non-daimi machine, use \texttt{make.py} instead of \texttt{MAKE}.

% This project depends on:

% \begin{itemize}
% \item SDL
% \item SDLImage
% \item GLEW
% \item boost
% \item cmake
% \end{itemize}

\subsection{Running}

Controlling the program is done using a set of inspector bars. By
default these are minimized, and show up as small squares in the lower
left corner. Clicking them will expand the inspector.

The keys \texttt{w,a,s,d} controls the placement of the camera, and
dragging the mouse with the left button down rotates it. Pressing
\texttt{1} or \texttt{2} selected weather to move the camera, or the
light source.

By default, shadow mapping is on, while ambient occlusion is
off. Ambient Occlusion can be enabled in the ambient occlusion
inspector.

\subsection{Files}
\label{sec:files}
A few notes about the many files in the project.

First a list of the most important directories:

\begin{center} 
  \begin{tabular}{|l|l|}
    \hline 
    \textbf{Directory} & \textbf{Description} \\ \hline 
    \texttt{src/} & The core parts of OpenEngine. \\ \hline
    \texttt{extensions/} & Extensions to OpenEngine, e.g. the OpenGL renderer. \\ \hline
    \texttt{projects/city/} & The files specific for this project. \\ \hline
  \end{tabular}
\end{center}

In the \texttt{projects/city/} directory, the following files are of interest:

\begin{center} 
  \begin{tabular}{|l|p{5cm}|}
    \hline 
    \textbf{File} & \textbf{Description} \\ \hline 
    \texttt{shaders/Shadow2.glsl.\{frag,vert\}} 
    & The fragment and vertex shader for the shadow map \\ \hline
    \texttt{Renderers/OpenGL/ShadowMap.cpp} 
    & This is the renderer, it extends the default OpenGL Renderer. (see note) \\ \hline    
    \texttt{Renderers/OpenGL/AmbientOcclusion.cpp} 
    & The ambient occlusion post-process module. \\ \hline    
    \texttt{shaders/surfaceNormals.glsl.\{frag,vert\}} 
    & The fragment and vertex shader for the surface normal shader \\ \hline
    \texttt{shaders/AmbientOcclusion.glsl.\{frag,vert\}} 
    & The fragment and vertex shader for the ambient occlusion shader \\ \hline
    \texttt{shaders/blurX.glsl.frag} 
    & A fragment shader for the gaussian blur kernel \\ \hline
    \texttt{shaders/blurY.glsl.frag} 
    & A fragment shader for the gaussian blur kernel \\ \hline
    \texttt{shaders/merge.glsl.frag} 
    & The fragment shader for the merge shader \\ \hline
  \end{tabular}
\end{center}

Note: the default renderer is in \\
\texttt{extensions/OpenGLRenderer/Renderers/OpenGL/Renderer.cpp}


%%% Local Variables: 
%%% mode: latex
%%% TeX-master: "master"
%%% End: 
